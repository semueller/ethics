\documentclass[10pt]{beamer}
\usetheme{metropolis}


\usepackage{appendixnumberbeamer}

\usepackage{booktabs}
\usepackage[scale=2]{ccicons}

\usepackage{pgfplots}
\usepgfplotslibrary{dateplot}

\usepackage{xspace}
\newcommand{\themename}{\textbf{\textsc{metropolis}}\xspace}

\title{Ethik}
\subtitle{in Forschung und Gesellschaft}
\date{\today}
\author{Luca Lach und Sebastian Mueller}
\institute{Aperture Science}

\begin{document}
		
	\maketitle
	
	\begin{frame}{Inhaltsverzeichnis}
		\setbeamertemplate{section in toc}[sections numbered]
		\tableofcontents[hideallsubsections]
	\end{frame}
	
\section{Definition Ethik}
\subsection{Entwicklung des Ethikbegriffs}
	%Moralphilosophie, Entstehung des Begriffs, Disziplinen, worauf wir uns konzentrieren werden
	%Wandelnde Weltanschauungen -> Wandelndes Verstaendnis der Ethik
	%Mittelalter: goettliche ordnung -> religioese Ethik
	%Neuzeit: Utilitarismus, Egoistische Ethik, 
	%Verankerung ethischer Normen in der modernen Gesellschaft -> Grundgesetz, Richtlinien, im Vortrag nun also insbesondere Richtlinien in der Forschung 
	%-> Rechte auf Selbstbestimmung, Gesundheit, persoenliche Entfaltung
\section{Ethische Richtlinien in der experimentellen Forschung}

\subsection{Motivation an kontroversem Experimenten}
%erste Diskussion was den Anwesenden auffaellt was da schief gelaufen sein koennte
% ein paar Konkrete Richtlinien ansprechen
\subsection{Richtlinien der APA}
%American Psychologie Association hat wegen diesen Kritiken folgende Richtlinien entwickelt
%
\subsection{Die Versuchsperson}
%freiwilliger Teilnehmer, moechte (idealerweise) der Forschung helfen
%Experiment
%Methodisch angelegte Untersuchung zur empirischen Gewinnung von Informationen. Ist eine kuenstliche Situation die vom Versuchsleiter geplant und herbeigefuehrtwird ueber die die VP dementsprechend nicht den vollen Ueberblick oder Kontrolle hat.
%daraus folgt besondere Verantwortung
\subsection{Wahrung der Selbstbestimmung}
%Aufklaerung, Vertrag schliessen
%Konflikt: Taeuschung von VP? Experiment Beispiel?

\subsection{Unwissende VP/ VP nicht ausnutzen}

\section{Design ethisch korrekter Experimente}
%was der Text dazu sagt

\subsection{Beispiele aus den Fachbereichen Robotik, VR}
%Sensibilisierung in Fachbereich

\subsection{Sensibilisierung: Ethik im Alltag(?)}
%Da Gesetze und Richtlinien nur die praktische Ausuebung unseres theoretischen Verstaendnisses von dem Ethikbegriff sind, sollten wir den Begriff nun auch in einem groesseren Kontext diskutieren um durch eine Einordnung in einen groesseren Kontext fuer das Thema zu sensibilisieren

%Zwangsrekrutierung Beispiel Google: Ist das ueberhaupt ein experiment? Warum fragen die nicht einfach ob man bei einem Beta-Test teilnehmen moechte?


%Verantwortung von uns als KI Forschern, wie gehen wir mit KI um?
%Arbeiten an moralisch fragwuerdigen Projekten oder erarbeiten von Wissen das bspw zur Kriegsfuehrung verwendet werden kann


\end{document}