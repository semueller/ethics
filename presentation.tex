\documentclass[10pt]{beamer}
\usetheme{metropolis}


\usepackage{appendixnumberbeamer}

\usepackage{booktabs}
\usepackage[scale=2]{ccicons}

\usepackage{pgfplots}
\usepgfplotslibrary{dateplot}

\usepackage{xspace}
\newcommand{\themename}{\textbf{\textsc{metropolis}}\xspace}

\title{Ethik}
\subtitle{in Forschung und Gesellschaft}
\date{\today}
\author{Luca Lach und Sebastian Mueller}
\institute{Aperture Science}

\begin{document}
		
	\maketitle
	
	\begin{frame}{Inhaltsverzeichnis}
		\setbeamertemplate{section in toc}[sections numbered]
		\tableofcontents[hideallsubsections]
	\end{frame}
	
\section{Definition Ethik}
\subsection{Entwicklung des Ethikbegriffs}
	%Moralphilosophie, Entstehung des Begriffs, Disziplinen, worauf wir uns konzentrieren werden
	%Wandelnde Weltanschauungen -> Wandelndes Verstaendnis der Ethik
	%Mittelalter: goettliche ordnung -> religioese Ethik
	%Neuzeit: Utilitarismus, Egoistische Ethik, 
	%Verankerung ethischer Normen in der modernen Gesellschaft -> Grundgesetz, Richtlinien, im Vortrag nun also insbesondere Richtlinien in der Forschung 
	%-> Rechte auf Selbstbestimmung, Gesundheit, persoenliche Entfaltung
\section{Ethische Richtlinien in der experimentellen Forschung}

\subsection{Motivation an kontroversem Experimenten}
%erste Diskussion was den Anwesenden auffaellt was da schief gelaufen sein koennte
% ein paar Konkrete Richtlinien ansprechen
\subsection{Richtlinien der APA}
%American Psychologie Association hat wegen diesen Kritiken folgende Richtlinien entwickelt
%
\subsection{Die Versuchsperson}
%freiwilliger Teilnehmer, moechte (idealerweise) der Forschung helfen
%Experiment
%Methodisch angelegte Untersuchung zur empirischen Gewinnung von Informationen. Ist eine kuenstliche Situation die vom Versuchsleiter geplant und herbeigefuehrtwird ueber die die VP dementsprechend nicht den vollen Ueberblick oder Kontrolle hat.
%daraus folgt besondere Verantwortung
\subsection{Wahrung der Selbstbestimmung}
%Aufklaerung, Vertrag schliessen
%Konflikt: Taeuschung von VP? Experiment Beispiel?

\subsection{Unwissende VP/ VP nicht ausnutzen}

\section{Design und Durchfuehrung ethisch korrekter Experimente}
%was der Text dazu sagt:
\subsection{Design}
%Grundsaeze: 1) Wahrung von Wuerde, Privatsphaere, psychische und physische Unversertheit; Ist die Gefaehrdung thematisch und/oder methodisch? Wie schwerwiegend wird die Beeintraechtigung eingeschaetzt? -> Voruntersuchung
%2) Herstellung von Transparenz; Teilnehmer vollstaendig ueber Untersuchungsgegenstand und Untersuchungsmaterialien informieren um ihnen eien informierte Entscheidung zu ermoeglichen ob sie bspw bei sensiblen Themen weiter Auskunft geben wollen
%3) Vermeidung von Taeuschungen;die Entscheidung muss man PERSOENLICH VERANTWORTEN KOENNEN WOW TOLL 
% Gespraeche mit anderen Wissenschaftlern [no shit shercock], alternative Untersuchungen planen (hierfuer vlt ein Beispiel durchsprechen? untersuchung die gegen ethische Richtlinien verstoesst so ummodellieren das sie ethisch in Ordnung ist)
\subsection{Durchfuehrung}
%Anwerben der VP -> keine zu grosse Belohnung anbieten?
%Aufklaerung -> offen und ehrlich sein, Vertrag schliessen! VP verpflichtet sich auch dazu die Untersuchung gewissenhaft zu absolvieren

\subsection{Beispiele aus den Fachbereichen Robotik, VR}
%Sensibilisierung in Fachbereich
% llachs BA
% semuellers Daene
% exposure therapy


\section{Diskussion: Ethik in der Informatik}
%Da Gesetze und Richtlinien nur die praktische Ausuebung unseres theoretischen Verstaendnisses von dem Ethikbegriff sind, sollten wir den Begriff nun auch in einem groesseren Kontext diskutieren um durch eine Einordnung in einen groesseren Kontext fuer das Thema zu sensibilisieren

%Zwangsrekrutierung Beispiel Google: Ist das ueberhaupt ein experiment? Warum fragen die nicht einfach ob man bei einem Beta-Test teilnehmen moechte?

%Verantwortung von uns als KI Forschern, wie gehen wir mit KI um?
%Arbeiten an moralisch fragwuerdigen Projekten oder erarbeiten von Wissen das bspw zur Kriegsfuehrung verwendet werden kann
% ai als produkt: Welche Verantwortung uebernehmen wir wenn wir Software vertreiben die komplexe/wichtige Aufgaben uebernimmt oder unterstuetzt deren Verhalten wir nicht in jeder Situation vorhersagen koennen. Wir programmieren ein komplexes Produkt das sich unvorhersehbar verhalten kann.
%Chirurgieroboter? Ueberwachungssoftware? Wie autonom darf diese Software sein? Wie sehr soll sich ein Mensch auf die Software verlassen duerfen? Was waeren ethisch korrekte Messgroessen um die beiden letzten Fragen beantworten zu koennen?

%Wie verhalten wir uns gegenueber KI?
% Minenroboter? Forscher hatten Mitleid beim Test von diesen Robotern; kann es zu einer schleichenden Verrohung kommen wenn wir humanoide Roboter ruecksichtslos "behandeln"?
%Selbstbewusste KI, welche Massstaebe wuerde man an die anlegen? Eigenen Willen respektieren? Man koennte sie einfach kopieren und mit der Kopie experimentieren und bei Schaeden danach loeschen?


\end{document}
